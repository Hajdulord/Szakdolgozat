\chapter{Bevezetés} % Introduction
\label{ch:intro}

\paragraph{}
A játékfejlesztésben, mint általánosságban a szoftverfejlesztésben nagy hangsúlyt kell fektetnünk a megfelelő tervezésbe hogy jól skálázható, egyszerűen átlátható kódot adjunk ki kezeink közül. Azonban a modern törekvések mint például a vizuális kódolás és a könnyen átlátható, valamint beletanulható keretrendszerek, és fejlesztői környezetek elterjedése miatt gyakran olyan emberek látnak hozzá nagy rendszerek, mi esetünkben játékok készítéséhez, akiknek nincs előzetes ismerete ilyenek megvalósításához. A bőségesen elérhető egyszerű eszközökkel hamar el lehet készíteni egyszerűbb azonban nehézkesen (néhány eseteben semmilyen módon sem) továbbfejleszthető játékokat. Ez azonban csak a jéghegy csúcsa. Rengeteg kezdő játékfejlesztő szimplán a játéka bejezéséig sem jut el. Legtöbb esetben ez azért történik, mert a kód átláthatatlan módosítások benne nehezen észrevehető mellékhatásokat okozhatnak, miknek javítása sok energiát követel. Egyszerűbb elölről kezdeni, más projektnek nekiállni, vagy egyszerűen feladni.
\paragraph{}
Szerintem azonban ezen problémák könnyedén kiküszöbölhetőek!\\*
Tervezési minták, régóta léteznek és fejlődnek a tradicionális szoftverfejlesztésben és a nagyiparos játékfejlesztésben is, azonban a független/kezdő játékfejlesztésben nem-régiekben kezd csak elterjedni, pedig meglátásom szerint néhány alapelvvel (mint például a S.O.L.I.D) összekapcsolva a feljebb kifejtett problémára tökéletes megoldást nyújt. Hogy pontosan hogy és miért az amit ez a Szakdolgozat fog megválaszolni, egy rövid 2D-s játékon keresztül.

