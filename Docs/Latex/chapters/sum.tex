\chapter{Összegzés} % Conclusion
\label{ch:sum}
A szakdolgozatom célja volt, hogy a teljesség igénye nélkül mutassak meg tervezési mintákat a játékfejlesztésben. Néhány mintára példát mutatni és szemléltetni, hogy miért növelik ezeknek a használata a szoftver minőségét és átláthatóságát. Saját meglátásom szerint ezt a célt elértem, mutattam tervezési mintákra felhasználási lehetőséget, kódbeli példákat, és további tervezési mintákat amiket más esetekben fel lehet használni. Végezetül egy olyan játékot készítettem, aminek értelmezés és továbbfejlesztése nem igényel felesleges erőfeszítést.

\section{Köszönetnyilvánítás}
Ezen szekcióba szeretném megköszönni Kovácsné Pusztai Kinga Tanárnőnek, hogy kérdéseimre, mindig hasznos információkkal szolgált és ezzel megfelelő irányba terelgette szakdolgozatom.\\
Köszönetemet szeretném nyilvánítani, Andi Petinek, Gyimesi Kristófnak, Szalai Patriknak, Yu Kai Tenak, Jakab Leventének és Orosz Ádámnak, hogy a félév során kipróbálták a játékomat, és szakmailag hasznos és értelmes visszajelzésekkel láttak el, jelezve nekem, hogy jó úton járok a szakdolgozatom készítése során.

