\subsubsection{Szakdolgozati Témabejelentő}
A szakdolgozatomban a modern játékfejlesztésben leggyakrabban használt tervezési mintákat mutatom be egy 2D-s játék fejlesztése
során. Ennek az a célja, hogy megmutassam, hogy a játékfejlesztésben, hogyan és pontosan milyen tervezési mintákat érdemes
alkalmazni. Mindezzel létrehozok, jól átlátható, könnyen bővíthető kódot. Ezen tényezők fontosságát azért tartom kimagaslónak,
mivel a mai játékfejlesztésnek és abba történő belépésnek a legnagyobb problémája a projektek be nem fejezése, ami legtöbbször az
előre meg nem tervezett, gyakran nem átlátható, nehézkesen bővíthető kód. Maga a 2D-s játék célja, hogy egy akadályokkal és
ellenfelekkel teli pályán végig navigáljunk és elérjük a végcélt. Ehhez az utóbbi években nagy népszerűségnek örvendő Unity
játékfejlesztői motort használnom, amiben komponens alapú szkripteket lehet létrehozni C\# nyelvben.
