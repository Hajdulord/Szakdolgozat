\chapter{Felhasználói dokumentáció} % User guide
\label{ch:user}

\section{A játék rövid leírása}
A játék Hack-and-Slash stílusú. A célunk, hogy haladjunk előre egy pályán, ahol különböző nehézségű és számosságú ellenfelek jönnek velünk szembe. A pályán találhatunk felvehető tárgyakat, amik szituációtól függően erősítenek bennünket. Halál esetén újraéledünk egy előre meghatározott ellenőrző ponton, ahonnan folytathatjuk a játékot. A pálya végén értékel minket a játék, és ha más játékosokhoz képest (mint egy árkád játékban, local high score) jobban teljesítettünk, vagy még nem telt be az eredménytábla, akkor az előbb említett táblára felkerülünk. A játék 3 pályát/szintet tartalmaz, amik egyre nehezebbek, ezen felül állásunkat el tudjuk mentei, valamint betölteni. Mindezzel egy kohézív gyors, kihívásokkal teli játékélményt kínálunk.
\subsection{Célközönség}
A játék azon eberek számára lehet érdekes, akik szeretik a gyors kihívással teli árkát stílusú Hack-and-Slash játékokat. 15 életévet minimum betöltött embernek ajánljuk, mivel a játék vért és erőszakot (ellenfelek legyőzése) tartalmaz.
\newpage
\section{Rendszerkövetelmények}
\begin{table}[htb]
	\centering
	\begin{tabular}{ | m{0.1\textwidth} | m{0.45\textwidth} | m{0.45\textwidth} | }
		\hline
		\textbf{Adatok} & \textbf{Minimum követelménye} & \textbf{Ajánlott követelmény} \\
		\hline \hline
		CPU & Intel i5-8250U & Intel i5-8600K \\ \hline
		GPU & Nvidia GeForce MX150 & Nvidia GTX 1050Ti \\ \hline
		RAM & 8Gb & 16Gb \\ \hline
		OS & \multicolumn{2}{c|}{Windows, Linux} \\
		\hline
		Disc & \multicolumn{2}{c|}{300Mb} \\
		\hline
	\end{tabular}
	\caption{rendszerkövetelmények}
	\label{sysReq}
\end{table}
\subsection{Játék indítása}
A játék nem igényel telepítést. Futtatásához indítsa el a 'The Quest for the Thesis.exe'-t (kattintson rá bal egérgombbal kétszer Windows-on).

\newpage
\section{Funkciók ismertetése}
A következő fejezetekben a játék funkcióinak részletesebb ismertetése lesz kifejtve. Hogyan is néz ki a játék, és mivel találkozhatunk miután elindítottuk.

\subsection{Grafikus Felhasználói Interfész}
Kettőféle Grafikus Felhasználói Interfész különböztethetünk meg a játék során. Az egyik, ami tisztán információt közöl velünk, a másik amivel interaktálhatunk is.
\subsubsection{Tisztán információ közlő GUI-k}
Játékon belüli indikátorok azok a grafikus elemek, amik visszajelzik nekünk a játékos aktuális állapotát. 
\begin{figure}[htb]
	\noindent\makebox[\textwidth]{
	\includegraphics[width= 1\textwidth]{inUse}}
	\caption{Játékon belüli indikátorok}
	\label{inUse}
\end{figure}

Itt látható a játékos jelenlegi élete.
\begin{figure}[htb]
	\noindent\makebox[\textwidth]{
	\includegraphics[width= 1\textwidth]{inUseHealth}}
	\caption{Játékos életereje}
	\label{inUseHealth}
\end{figure}

A játékos képességei és tárgyai, valamint az idő addig amíg újra lehet használni a tárgyat/képességet.
\begin{figure}[htb]
	\noindent\makebox[\textwidth]{
	\includegraphics[width= 1\textwidth]{inUseItems}}
	\caption{Játékos képességei és tárgyai}
	\label{inUseItems}
\end{figure}

Valamint a játékost éppen érintő státuszeffekteket.
\begin{figure}[htb]
	\noindent\makebox[\textwidth]{
	\includegraphics[width= 1\textwidth]{inUseStatuses}}
	\caption{Játékost érintő státuszeffekteket}
	\label{inUseStatuses}
\end{figure}

Ezeken felül ebbe a kategóriába tartozik még a töltőképernyő, ami a szintek között jelenhet meg.
\begin{figure}[H]
	\noindent\makebox[\textwidth]{
	\includegraphics[width= 1\textwidth]{loading}}
	\caption{Töltőképernyő}
	\label{loading}
\end{figure}

\newpage
Valamint a halálképernyő, ami akkor jelenik meg, ha a játékos meghalt.
\begin{figure}[H]
	\noindent\makebox[\textwidth]{
	\includegraphics[width= 1\textwidth]{dead}}
	\caption{Halálképernyő}
	\label{dead}
\end{figure}

\subsubsection{Interaktív GUI-k}

\subsection{Pályák}
\subsection{Irányítás}
\subsection{Tárgyak}
\subsection{Ellenfelek}