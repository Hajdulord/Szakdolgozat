\chapter{Felhasználói dokumentáció} % User guide
\label{ch:user}

\section{A játék rövid leírása}
A játék Hack-and-Slash stílusú. A célunk, hogy haladjunk előre egy pályán, ahol különböző nehézségű és számosságú ellenfelek jönnek velünk szembe. A pályán találhatunk felvehető tárgyakat, amik szituációtól függően erősítenek bennünket. Halál esetén újraéledünk egy előre meghatározott ellenőrző ponton, ahonnan folytathatjuk a játékot. A pálya végén értékel minket a játék, és ha más játékosokhoz képest (mint egy árkád játékban, local high score) jobban teljesítettünk, vagy még nem telt be az eredménytábla, akkor az előbb említett táblára felkerülünk. A játék 3 pályát/szintet tartalmaz, amik egyre nehezebbek, ezen felül állásunkat el tudjuk mentei, valamint betölteni. Mindezzel egy kohézív gyors, kihívásokkal teli játékélményt kínálunk.
\subsection{Célközönség}
A játék azon eberek számára lehet érdekes, akik szeretik a gyors kihívással teli árkát stílusú Hack-and-Slash játékokat. 15 életévet minimum betöltött embernek ajánljuk, mivel a játék vért és erőszakot (ellenfelek legyőzése) tartalmaz.
\cleardoublepage
\section{Rendszerkövetelmények}
\begin{table}[htb]
	\centering
	\begin{tabular}{ | m{0.1\textwidth} | m{0.45\textwidth} | m{0.45\textwidth} | }
		\hline
		\textbf{Adatok} & \textbf{Minimum követelménye} & \textbf{Ajánlott követelmény} \\
		\hline \hline
		CPU & Intel i5-8250U & Intel i5-8600K \\ \hline
		GPU & Nvidia GeForce MX150 & Nvidia GTX 1050Ti \\ \hline
		RAM & 8Gb & 16Gb \\ \hline
		OS & \multicolumn{2}{c|}{Windows, Linux} \\
		\hline
		Disc & \multicolumn{2}{c|}{300Mb} \\
		\hline
	\end{tabular}
	\caption{rendszerkövetelmények}
	\label{sysReq}
\end{table}
\subsection{Játék indítása}
A játék nem igényel telepítést. Futtatásához indítsa el a 'The Quest for the Thesis.exe'-t (kattintson rá bal egérgombbal kétszer Windows-on).

\cleardoublepage
\section{Funkciók ismertetése}
A következő fejezetekben a játék funkcióinak részletesebb ismertetése lesz kifejtve. Hogyan is néz ki a játék, és mivel találkozhatunk miután elindítottuk.

\subsection{Grafikus Felhasználói Interfész (GUI)}
Kettőféle Grafikus Felhasználói Interfész különböztethetünk meg a játék során. Az egyik, ami tisztán információt közöl velünk, a másik amivel interaktálhatunk is.
\subsubsection{Tisztán információ közlő GUI-k}
Játékon belüli indikátorok azok a grafikus elemek, amik visszajelzik nekünk a játékos aktuális állapotát. 
\begin{figure}[htb]
	\noindent\makebox[\textwidth]{
	\includegraphics[width= 1\textwidth]{inUse}}
	\caption{Játékon belüli indikátorok}
	\label{inUse}
\end{figure}

Itt látható a játékos jelenlegi élete.
\begin{figure}[htb]
	\noindent\makebox[\textwidth]{
	\includegraphics[width= 1\textwidth]{inUseHealth}}
	\caption{Játékos életereje}
	\label{inUseHealth}
\end{figure}

A játékos képességei és tárgyai, valamint az idő addig amíg újra lehet használni a tárgyat/képességet.
\begin{figure}[htb]
	\noindent\makebox[\textwidth]{
	\includegraphics[width= 1\textwidth]{inUseItems}}
	\caption{Játékos képességei és tárgyai}
	\label{inUseItems}
\end{figure}

Valamint a játékost éppen érintő státuszeffekteket.
\begin{figure}[htb]
	\noindent\makebox[\textwidth]{
	\includegraphics[width= 1\textwidth]{inUseStatuses}}
	\caption{Játékost érintő státuszeffekteket}
	\label{inUseStatuses}
\end{figure}

Ezeken felül ebbe a kategóriába tartozik még a töltőképernyő, ami a szintek között jelenhet meg.
\begin{figure}[H]
	\noindent\makebox[\textwidth]{
	\includegraphics[width= 1\textwidth]{loading}}
	\caption{Töltőképernyő}
	\label{loading}
\end{figure}

\cleardoublepage
Valamint a halálképernyő, ami akkor jelenik meg, ha a játékos meghalt.
\begin{figure}[H]
	\noindent\makebox[\textwidth]{
	\includegraphics[width= 1\textwidth]{dead}}
	\caption{Halálképernyő}
	\label{dead}
\end{figure}

\subsubsection{Interaktív GUI-k}
Ezeken a képernyőkön vannak gombok, lenyíló listák, vagy csúszkák, amikkel interaktálhatunk a játékkal.

\begin{enumerate}
	\item \label{fomenu} \textbf{Főmenü} az első GUI elem amit látunk a játék indításakor. Öt gomb látható ezen a felületen:
	\begin{itemize}
		\item \textit{Start} ha erre a gombra kattintunk akkor tovább lépünk a \textit{Pályaválasztóra \ref{levelSelector}}.
		\item \textit{Settings} erre kattintva megnyitjuk a \textit{Beállítások \ref{setting}} menüt.
		\item \textit{Score Board}-ra kattintva eljutunk az \textit{Eredménytáblára \ref{scoreboard}}.
		\item \textit{Credit} gombra kattintva eljutunk a \textit{Köszönetnyilvánító \ref{credit}} menüre.
		\item \textit{Exit} gomb lenyomásával kilépünk a játékból.
	\end{itemize}
	\begin{figure}[H]
		\noindent\makebox[\textwidth]{
		\includegraphics[width= 1\textwidth]{mainMenu}}
		\caption{A Kezdőképernyő}
		\label{mainMenu}
	\end{figure}
	
	\item \label{levelSelector} \textbf{Pályaválasztó} menün  adhatjuk meg a nevünket ami később az \textit{Eredménytáblán \ref{scoreboard}} látható.
	\begin{itemize}
	\item A \textit{Easy, Medium} és \textit{Hard} gombok lenyomásával átléphetünk az \textit{Utasítások \ref{instructions}} menübe, és a játék betölti a megfelelő pályát.
	\item \textit{Load} gomb lenyomását követően átlépünk a \textit{Mentés és Betöltés \ref{save}} menübe.
	\end{itemize}
	\begin{figure}[H]
		\noindent\makebox[\textwidth]{
		\includegraphics[width= 1\textwidth]{levelSelection}}
		\caption{A Pályaválasztó menü}
		\label{levelSelection}
	\end{figure}
	
	\item \label{instructions} \textbf{Utasítások} menüben láthatók a játék bemeneteire végrehajtott akciók.
	\begin{itemize}
		\item \textit{Play} gomb lenyomását követően elkezdődik a játék.
	\end{itemize}
	\begin{figure}[H]
		\noindent\makebox[\textwidth]{
		\includegraphics[width= 1\textwidth]{instructions}}
		\caption{Utasítások menü}
		\label{instructionsF}
	\end{figure}	
	
	\item \label{scoreboard} \textbf{Eredménytábla} ahol a szinteken elért eredmények láthatók.
	\begin{itemize}
		\item \textit{Back} gomb lenyomását követően visszatérünk az előző menübe.
	\end{itemize}
	\begin{figure}[H]
		\noindent\makebox[\textwidth]{
		\includegraphics[width= 1\textwidth]{scoreBoard}}
		\caption{Eredménytábla}
		\label{scoreBoardF}
	\end{figure}
	
	\item \label{credit} \textbf{Köszönetnyilvánító} menün található a játék készítéséhez felhasznált assetek készítői.
	\begin{itemize}
		\item \textit{Back} gomb lenyomását követően visszatérünk az előző menübe.
	\end{itemize}
	\begin{figure}[H]
		\noindent\makebox[\textwidth]{
		\includegraphics[width= 1\textwidth]{credit}}
		\caption{Köszönetnyilvánító menü}
		\label{creditF}
	\end{figure}
	
	\item \label{setting} \textbf{Beállítások} menüben változtathatjuk meg az alapértelmezett beállításainkat.
	\begin{itemize}
		\item Bal felül választhatjuk ki egy lenyíló listából a kívánt felbontást.
		\item Jobb felül választhatjuk ki egy lenyíló listából a játékablak üzemmódját.
		\item \textit{Graphics Quality} menüpontban egy lenyíló listából kiválaszthatjuk a játék minőségi beállításait.
		\item \textit{Main Volume} csúszkán a fő hangerőt tudjuk állítani.
		\item \textit{Music Volume} csúszkán a zene hangerejét tudjuk állítani.
		\item \textit{SFX Volume} csúszkán az effektek hangerejét tudjuk állítani.
		\item \textit{Back} gomb lenyomását követően visszatérünk az előző menübe.
	\end{itemize}
	\begin{figure}[H]
		\noindent\makebox[\textwidth]{
		\includegraphics[width= 1\textwidth]{settings}}
		\caption{Beállítások}
		\label{settings}
	\end{figure}
	
	\item \label{save} \textbf{Mentés és Betöltés} menün található a 4 mentési hely. Ahol ha már volt mentésünk akkor a mentésben szereplő adatok jelennek meg, ha nem akkor az \textit{Empty Slot} felirat.   
	\begin{itemize}
		\item \textit{Load} gombot megnyomva a kijelölt mentés fog betölteni.
		\item \textit{Save} gombra kattintva a kijelölt mentési helyre elmentődik a jelenlegi állás (csak játék közben látható).
		\item \textit{Back} gomb lenyomását követően visszatérünk az előző menübe.
	\end{itemize}
	\begin{figure}[H]
		\noindent\makebox[\textwidth]{
		\includegraphics[width= 1\textwidth]{saveLoad}}
		\caption{Mentés és Betöltés menü}
		\label{saveLoad}
	\end{figure}
	
	\item \label{pause} \textbf{Szünet} menüt a játék közben az \textbf{ESC} gomb lenyomásával hívhatjuk elő.
	\begin{itemize}
		\item \textit{Resume} gomb lenyomását követően a játék folytatódik.
		\item \textit{Save / Load} gombot megnyomva továbblépünk a \textit{Mentés és Betöltés \ref{save}} menüre.
		\item \textit{Exit} gomb lenyomását követően a játék bezáródik.
	\end{itemize}
	\begin{figure}[H]
		\noindent\makebox[\textwidth]{
		\includegraphics[width= 1\textwidth]{pause}}
		\caption{Szünet menü}
		\label{pauseF}
	\end{figure}
	
	\item \label{end} \textbf{Vég} menü egy pálya végén jelenik meg.
	\begin{itemize}
		\item Legfelül látható a pálya elvégzését követő eredményünk.
		\item A szerzett pontunk alatt láthatóak a többi pályákat reprezentáló gombok, amik re kattintva a megfelelő pálya betöltődik (különböző pályákról érkezve a Vég menübe más pályák jelennek meg itt).
		\item \textit{Load} gombot megnyomva továbblépünk a \textit{Mentés és Betöltés \ref{save}} menüre.
		\item \textit{Score Board}-ra kattintva eljutunk az \textit{Eredménytáblára \ref{scoreboard}}.
		\item \textit{Credit} gombra kattintva eljutunk a \textit{Köszönetnyilvánító \ref{credit}} menüre.
		\item \textit{Exit} gomb lenyomását követően a játék bezáródik.
	\end{itemize}
	\begin{figure}[H]
		\noindent\makebox[\textwidth]{
		\includegraphics[width= 1\textwidth]{end}}
		\caption{Vég menü}
		\label{endF}
	\end{figure}
	
\end{enumerate}

\subsection{Pályák}
A játékban három pályát különítünk el: Easy, Medium, Hard

\begin{itemize}
	\item \textbf{Easy} pálya a névből eredően a legkönnyebb. A tárgyak és az ellenfelek megismerésére tökéletesen alkalmas.
	\begin{figure}[H]
		\noindent\makebox[\textwidth]{
		\includegraphics[width= 1\textwidth]{easy}}
		\caption{Easy pálya}
	\end{figure}
		\cleardoublepage
		\item \textbf{Medium} pálya közepes nehézségű. Több ellenség érkezik felén és a felvehető tárgyak ritkák.
	\begin{figure}[H]
		\noindent\makebox[\textwidth]{
		\includegraphics[width= 1\textwidth]{medium}}
		\caption{Medium pálya}
	\end{figure}
	
		\item \textbf{Hard} pálya a legnehezebb. A tárgyak kevesek, az ellenségek számosak és az erősebb fajtából valók.
	\begin{figure}[H]
		\noindent\makebox[\textwidth]{
		\includegraphics[width= 1\textwidth]{hard}}
		\caption{Hard pálya}
	\end{figure}
\end{itemize}

\subsection{Irányítás}
\begin{table}[H]
	\centering
	\begin{tabular}{ | m{0.3\textwidth} | m{0.5\textwidth} | }
		\hline
		\textbf{Akció} & \textbf{Gomb} \\
		\hline \hline
		Bal irányú mozgás & A \\ \hline
		Jobb irányú mozgás & D \\ \hline
		Ugrás & SPACE \\ \hline
		Gyors hirtelen mozgás & Baloldali ALT \\ \hline
		Sima Támadás & Baloldali egérgomb \\ \hline
		1, 2, 3, 4 gombok & megfelelő helyen lévő tárgy használata \\ \hline
		Tárgyak felvétele & E \\ \hline
		Játék megállítása & ESC \\ \hline
	\end{tabular}
	\label{instruc}
\end{table}

\subsection{Tárgyak}
\begin{table}[H]
	\centering
	\begin{tabular}{ | m{0.2\textwidth} | m{0.2\textwidth} | m{0.5\textwidth} | N |}
		\hline
		\textbf{Név} & \textbf{Kép} & \textbf{Leírás} & \\
		\hline \hline
		Gyógyító ital & 
			\noindent\makebox[0.2\textwidth]{
			\includegraphics[width= 0.2\textwidth]{CurePotion}}
		& Megszünteti az összes státuszeffektust.
		& \label{que:curePotion}
		\\ \hline
		
		Életerő ital & 
			\noindent\makebox[0.2\textwidth]{
			\includegraphics[width= 0.2\textwidth]{HealthPotion}}
		& Gyógyítja a játékost és a \textit{Gyógyul \ref{que:healing}} státuszeffektet rakja rá.
		& \label{que:healthPotion}
		\\ \hline		
		
		Tűz varázsfókusz & 
			\noindent\makebox[0.2\textwidth]{
			\includegraphics[width= 0.2\textwidth]{FireMagicFocus}}
		& Lehetővé teszi a \textit{Tűzkarika \ref{que:fireBurst}} varázslat használatára a játékost.
		& \label{que:fireMagicFocus}
		\\ \hline
		
		Jég varázsfókusz & 
			\noindent\makebox[0.2\textwidth]{
			\includegraphics[width= 0.2\textwidth]{IceMagicFocus}}
		& Lehetővé teszi a \textit{Jéglándzsa \ref{que:iceLance}} varázslat használatára a játékost.
		& \label{que:iceMagicFocus}
		\\ \hline
		
		Katana & 
			\noindent\makebox[0.2\textwidth]{
			\includegraphics[width= 0.2\textwidth]{Katana}}
		& Egy egyszerű kard amivel sebzed az ellenségeket.
		& \label{que:katana}
		\\ \hline
		
		Masamune & 
			\noindent\makebox[0.2\textwidth]{
			\includegraphics[width= 0.2\textwidth]{Masamune}}
		& Gyengén sebzi az ellenségeket, azonban \textit{Vérzés \ref{que:bleeding}} státuszeffektet rak rájuk.
		& \label{que:masamune}
		\\ \hline
		
		Muramasa & 
			\noindent\makebox[0.2\textwidth]{
			\includegraphics[width= 0.2\textwidth]{Muramasa}}
		& Erősen megsebzi az ellenségeket, és \textit{Kábult \ref{que::stunned}} státuszeffektet rak rájuk, azonban alacsony a támadási sebessége.
		& \label{que:muramasa}
		\\ \hline
		
	\end{tabular}
\end{table}

\subsection{Varázslatok}
\begin{table}[H]
	\centering
	\begin{tabular}{| m{0.2\textwidth} | m{0.2\textwidth} | m{0.5\textwidth} | N |}
		\hline
		\textbf{Név} & \textbf{Kép} & \textbf{Leírás} & \\
		\hline \hline
		
		Tűzkarika & 
			\noindent\makebox[0.2\textwidth]{
			\includegraphics[width= 0.2\textwidth]{fireBurst}}
		& Rengeteg sebzést okoz az összes közeli ellenségnek, és az \textit{Égés \ref{que:burning}} státuszeffektet rakja rájuk, nagy az újrahasználási ideje.
		& \label{que:fireBurst}
		\\ \hline
		
		Jéglándzsa & 
			\noindent\makebox[0.2\textwidth]{
			\includegraphics[width= 0.2\textwidth]{iceLance}}
		& Megsebzi az ellenségeket, és \textit{Fagyás \ref{que:frozen}} státuszeffektet rak rájuk.
		& \label{que:iceLance}
		\\ \hline
		
	\end{tabular}
\end{table}

\subsection{Státuszeffektusok}

\begin{table}[H]
	\centering
	\begin{tabular}{| m{0.2\textwidth} | m{0.2\textwidth} | m{0.5\textwidth} | N |}
		\hline
		 \textbf{Név} & \textbf{Kép} & \textbf{Leírás} &\\
		\hline \hline
		
		Vérzés & 
			\noindent\makebox[0.15\textwidth]{
			\includegraphics[width= 0.15\textwidth]{Bleeding}}
		& Időközönként megsebzi kicsit a célpontot.
		& \label{que:bleeding}
		\\ \hline
		
		Égés & 
			\noindent\makebox[0.15\textwidth]{
			\includegraphics[width= 0.15\textwidth]{Burning}}
		& Időközönként megsebzi a célpontot.
		& \label{que:burning}
		\\ \hline
		
		Fagyás & 
			\noindent\makebox[0.15\textwidth]{
			\includegraphics[width= 0.15\textwidth]{Frozen}}
		& Időközönként megsebzi és lelassítja a célpontot.
		& \label{que:frozen}
		\\ \hline
		
		Kábult & 
			\noindent\makebox[0.15\textwidth]{
			\includegraphics[width= 0.15\textwidth]{Stunned}}
		& Rendkívül megsebzi a célpontot és lezárja a mozgási lehetőségeit egy időre.
		& \label{que::stunned}
		\\ \hline
		
		Gyógyul & 
			\noindent\makebox[0.15\textwidth]{
			\includegraphics[width= 0.15\textwidth]{Healing}}
		& Időközönként gyógyítja a célpontot.
		& \label{que:healing}
		\\ \hline		
		
	\end{tabular}
\end{table}

\cleardoublepage

\subsection{Ellenfelek}
Minden ellenségnek kettő típusa van: járőr és őr. Az őr egy helybe áll és vár a játékosra. A járőr kettő pont között mozog.
\begin{center}
	\begin{longtable}{| m{0.2\textwidth} | m{0.2\textwidth} | m{0.5\textwidth} | N |}
	%\centering
		\hline
		\textbf{Név} & \textbf{Kép} & \textbf{Leírás} & \\
		\hline \hline
		
		Alap Ellenség & 
			\noindent\makebox[0.2\textwidth]{
			\includegraphics[width= 0.2\textwidth]{enemy}}
		& A leggyengébb ellenség, ami egy \textit{Katanaát \ref{que:katana}} használ a fő fegyvereként.
		& \label{que:enemy}
		\\ \hline
		
		Erős Ellenség & 
			\noindent\makebox[0.2\textwidth]{
			\includegraphics[width= 0.2\textwidth]{strongEnemy}}
		& Egy lényegesen erősebb ellenség, ami egy \textit{Masamunet \ref{que:masamune}} használ a fő fegyvereként.
		& \label{que:strongEnemy}
		\\ \hline		
		
		Tűzvarázsló & 
			\noindent\makebox[0.2\textwidth]{
			\includegraphics[width= 0.2\textwidth]{fireMage}}
		& Egy varázsló egység, ami egy \textit{Muramasat \ref{que:muramasa}} használ a fő fegyvereként, azonban képes a \textit{Tűzgyűrű \ref{que:muramasa}} varázslat használatára is.
		& \label{que:fireMage}
		\\ \hline
		
		Jégvarázsló & 
			\noindent\makebox[0.2\textwidth]{
			\includegraphics[width= 0.2\textwidth]{iceMage}}
		& Egy rendkívül veszélyes varázsló egység, ami egy \textit{Masamunet \ref{que:masamune}} használ a fő fegyvereként és képes a \textit{Jéglándzsa \ref{que:muramasa}} varázslat használatára is.
		& \label{que:iceeMage}
		\\ \hline		
		
	\end{longtable}
\end{center}